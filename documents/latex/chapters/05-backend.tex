% ============================================================
% Chapitre 5 : Architecture Backend
% ============================================================

\chapter{Architecture Backend}
\label{chap:backend}

\section{Vue d'Ensemble}

Le backend ProAct est développé avec \textbf{FastAPI}, un framework Python moderne offrant des performances élevées et une documentation automatique. L'architecture suit les principes de \textbf{Clean Architecture}.

\begin{techbox}[Stack Backend]
\begin{itemize}
    \item \textbf{Framework} : FastAPI 0.100+
    \item \textbf{ORM} : SQLAlchemy 2.0
    \item \textbf{Validation} : Pydantic v2
    \item \textbf{Auth} : python-jose (JWT)
    \item \textbf{ML} : Scikit-learn, XGBoost, Prophet
    \item \textbf{LLM} : Ollama
    \item \textbf{Vector Store} : FAISS
\end{itemize}
\end{techbox}

\section{Structure du Projet}

Le backend suit une organisation modulaire :

\begin{itemize}
    \item \file{app/main.py} : Point d'entrée FastAPI, configuration CORS
    \item \file{app/database.py} : Configuration SQLAlchemy
    \item \file{app/models.py} : Modèles ORM
    \item \file{app/schemas.py} : Schémas Pydantic
    \item \file{app/security.py} : Auth JWT et guards RBAC
    \item \file{app/routers/} : 15 routers par domaine
    \item \file{app/services/} : Logique métier
\end{itemize}

\section{Couche Routers}

Les routers définissent les endpoints REST :

\begin{table}[H]
\centering
\caption{Routers principaux}
\label{tab:routers}
\small
\begin{tabularx}{\textwidth}{@{}l l X@{}}
\toprule
\textbf{Router} & \textbf{Préfixe} & \textbf{Fonctionnalités} \\
\midrule
equipment & /api/equipment & CRUD équipements, stats \\
interventions & /api/interventions & Ordres de travail \\
spare\_parts & /api/spare-parts & Inventaire, alertes \\
technicians & /api/technicians & Profils, compétences \\
kpi & /api/kpi & MTBF, MTTR, Dashboard \\
amdec & /api/amdec & Modes défaillance, RPN \\
rag & /api/rag & Upload docs, requêtes \\
copilot & /api/copilot & Assistant IA \\
ocr & /api/ocr & Extraction texte \\
prediction & /api/predict & Prévision pannes \\
\bottomrule
\end{tabularx}
\end{table}

\section{Couche Services}

Les services encapsulent la logique métier :

\begin{itemize}
    \item \textbf{KPIService} : Calculs MTBF, MTTR, Disponibilité
    \item \textbf{CopilotService} : Orchestration des requêtes IA
    \item \textbf{PredictionService} : Modèles ML pour RUL
    \item \textbf{OCRService} : Extraction via modèles VLM
    \item \textbf{RAGService} : Pipeline de recherche documentaire
\end{itemize}

\section{Gestion des Erreurs}

\begin{table}[H]
\centering
\caption{Codes d'erreur HTTP}
\label{tab:http-errors}
\small
\begin{tabularx}{\textwidth}{@{}l l X@{}}
\toprule
\textbf{Code} & \textbf{Signification} & \textbf{Exemple} \\
\midrule
400 & Bad Request & Validation Pydantic échouée \\
401 & Unauthorized & JWT expiré ou invalide \\
403 & Forbidden & Rôle non autorisé \\
404 & Not Found & Équipement ID inconnu \\
422 & Unprocessable & Champ requis manquant \\
500 & Server Error & Exception non gérée \\
\bottomrule
\end{tabularx}
\end{table}

\section{Endpoints de Santé}

Le backend expose des endpoints de monitoring :

\begin{itemize}
    \item \api{GET /health} : Vérification DB et RAG
    \item \api{GET /api/stats} : Compteurs des entités
    \item \api{GET /} : Informations version
\end{itemize}

\section{Documentation API}

FastAPI génère automatiquement la documentation :

\begin{itemize}
    \item \textbf{Swagger UI} : \api{/docs}
    \item \textbf{ReDoc} : \api{/redoc}
    \item \textbf{OpenAPI} : \api{/openapi.json}
\end{itemize}

\begin{infobox}[Tags API]
Les endpoints sont organisés par tags : Equipment, Interventions, Spare Parts, Technicians, KPIs, AMDEC, RAG, Copilot, OCR, Forecast.
\end{infobox}

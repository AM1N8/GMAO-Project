% ============================================================
% Chapitre 4 : Authentification et Autorisation
% ============================================================

\chapter{Authentification et Autorisation}
\label{chap:auth}

\section{Architecture de Sécurité}

ProAct implémente une architecture d'authentification moderne basée sur \textbf{JWT} (JSON Web Tokens) avec \textbf{Supabase Auth} comme fournisseur d'identité. Le système adopte une approche de \textbf{single source of truth} où les rôles utilisateurs sont stockés dans les métadonnées Supabase.

\begin{figure}[H]
\centering
\begin{tikzpicture}[node distance=1cm and 1.2cm]
    % Nodes
    \node[component, minimum width=2.5cm] (user) {Utilisateur};
    \node[component, right=1.5cm of user, fill=secondary!15, minimum width=2.5cm] (supabase) {Supabase};
    \node[component, below=1.2cm of supabase, fill=accent!15, minimum width=2.5cm] (frontend) {Frontend};
    \node[component, below=1.2cm of frontend, fill=primary!15, minimum width=2.5cm] (backend) {Backend};
    
    % Arrows
    \draw[arrow] (user) -- node[above, font=\footnotesize] {1. Login} (supabase);
    \draw[arrow] (supabase) -- node[right, font=\footnotesize] {2. JWT} (frontend);
    \draw[arrow] (frontend) -- node[right, font=\footnotesize] {3. API} (backend);
    \draw[arrow, dashed] (backend.east) -- ++(1,0) |- node[right, font=\footnotesize, pos=0.25] {4. Verify} (supabase.east);
\end{tikzpicture}
\caption{Flux d'authentification}
\label{fig:auth-flow}
\end{figure}

\section{Intégration Makerkit}

Le boilerplate Makerkit fournit une intégration Supabase clé en main incluant :

\begin{itemize}
    \item Configuration client Supabase (browser et server)
    \item Pages d'authentification pré-construites
    \item Middleware de protection des routes
    \item Gestion des sessions avec refresh automatique
\end{itemize}

\section{Modèle de Données Utilisateur}

Le backend définit une dataclass \tech{AuthUser} avec :

\begin{itemize}
    \item \textbf{id} : Identifiant Supabase unique (claim \tech{sub})
    \item \textbf{email} : Adresse email de l'utilisateur
    \item \textbf{role} : Rôle extrait de \tech{app\_metadata}
    \item \textbf{raw\_claims} : Claims JWT bruts
\end{itemize}

Le système définit quatre rôles hiérarchiques :

\begin{table}[H]
\centering
\caption{Hiérarchie des rôles}
\label{tab:roles-enum}
\small
\begin{tabularx}{\textwidth}{@{}l l X@{}}
\toprule
\textbf{Rôle} & \textbf{Valeur} & \textbf{Description} \\
\midrule
\role{Admin} & admin & Accès total, gestion système \\
\role{Supervisor} & supervisor & Gestion opérationnelle \\
\role{Technician} & technician & Exécution interventions \\
\role{Viewer} & viewer & Consultation seule \\
\bottomrule
\end{tabularx}
\end{table}

\section{Vérification JWT}

Le processus de vérification comprend :

\begin{enumerate}
    \item Extraction du token depuis l'en-tête \tech{Authorization: Bearer <token>}
    \item Vérification de la signature avec \tech{SUPABASE\_JWT\_SECRET}
    \item Validation de l'expiration et des claims requis
    \item Extraction du rôle depuis \tech{app\_metadata.role}
\end{enumerate}

\section{Contrôle d'Accès (RBAC)}

Le système fournit une factory \tech{require\_role()} qui crée des dépendances FastAPI pour le contrôle d'accès. Des raccourcis sont disponibles : \tech{require\_admin()}, \tech{require\_supervisor\_or\_admin()}, \tech{require\_technician\_or\_above()}.

\section{Matrice des Permissions}

\begin{table}[H]
\centering
\caption{Matrice des permissions par endpoint}
\label{tab:permissions-matrix}
\small
\begin{tabularx}{\textwidth}{@{}X c c c c@{}}
\toprule
\textbf{Endpoint} & \textbf{Adm} & \textbf{Sup} & \textbf{Tech} & \textbf{View} \\
\midrule
\multicolumn{5}{l}{\textit{Équipements}} \\
GET /equipment & \checkmark & \checkmark & \checkmark & \checkmark \\
POST /equipment & \checkmark & \checkmark & -- & -- \\
DELETE /equipment/:id & \checkmark & -- & -- & -- \\
\midrule
\multicolumn{5}{l}{\textit{Interventions}} \\
GET /interventions & \checkmark & \checkmark & \checkmark & \checkmark \\
POST /interventions & \checkmark & \checkmark & \checkmark & -- \\
\midrule
\multicolumn{5}{l}{\textit{IA}} \\
POST /rag/query & \checkmark & \checkmark & \checkmark & \checkmark \\
POST /ocr/extract & \checkmark & \checkmark & -- & -- \\
\bottomrule
\end{tabularx}
\end{table}

\section{Sécurité des Sessions}

\begin{warningbox}[Bonnes Pratiques Implémentées]
\begin{itemize}
    \item Tokens JWT signés avec secret cryptographique (256 bits min)
    \item Expiration des tokens configurée (par défaut 1 heure)
    \item Refresh tokens pour renouvellement transparent
    \item Validation stricte de tous les claims JWT requis
    \item Logging des tentatives d'accès non autorisées
\end{itemize}
\end{warningbox}

\section{Gestion des Erreurs}

\begin{table}[H]
\centering
\caption{Codes d'erreur d'authentification}
\label{tab:auth-errors}
\small
\begin{tabularx}{\textwidth}{@{}l l X@{}}
\toprule
\textbf{Code} & \textbf{Situation} & \textbf{Action} \\
\midrule
401 Unauthorized & Token manquant/invalide & Redirection vers login \\
403 Forbidden & Rôle insuffisant & Affichage erreur \\
500 Server Error & JWT\_SECRET non configuré & Alerte admin \\
\bottomrule
\end{tabularx}
\end{table}

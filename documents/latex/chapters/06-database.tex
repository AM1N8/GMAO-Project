% ============================================================
% Chapitre 6 : Modèle de Données
% ============================================================

\chapter{Modèle de Données}
\label{chap:database}

\section{Vue d'Ensemble}

Le modèle de données de ProAct est implémenté avec \textbf{SQLAlchemy ORM} et conçu pour capturer l'ensemble des entités et relations d'un système GMAO complet. La base de données supporte PostgreSQL (production) et SQLite (développement).

\section{Schéma Entités-Relations}

\begin{figure}[H]
\centering
\begin{tikzpicture}[
    node distance=1.5cm and 2.5cm,
    entity/.style={
        rectangle,
        draw=primary,
        fill=primary!10,
        minimum width=2.5cm,
        minimum height=0.8cm,
        font=\small\bfseries
    }
]
    % Main entities
    \node[entity] (equipment) {Equipment};
    \node[entity, right=2.5cm of equipment] (intervention) {Intervention};
    \node[entity, below=1.2cm of equipment] (technician) {Technician};
    \node[entity, below=1.2cm of intervention] (sparepart) {SparePart};
    
    % Secondary entities
    \node[entity, above=1cm of equipment, fill=warning!10, draw=warning, minimum width=2cm, font=\footnotesize\bfseries] (failuremode) {FailureMode};
    \node[entity, right=1cm of failuremode, fill=warning!10, draw=warning, minimum width=2cm, font=\footnotesize\bfseries] (rpn) {RPNAnalysis};
    
    % Relationships
    \draw[arrow] (equipment) -- node[above, font=\footnotesize] {1:N} (intervention);
    \draw[arrow] (equipment) -- node[left, font=\footnotesize] {1:N} (failuremode);
    \draw[arrow] (failuremode) -- node[above, font=\footnotesize] {1:N} (rpn);
    \draw[arrow] (intervention.south) -- ++(0,-0.4) -| ([xshift=-2mm]technician.east);
    \draw[arrow] (intervention.south) -- ++(0,-0.4) -| ([xshift= 2mm]sparepart.west);
\end{tikzpicture}
\caption{Diagramme entités-relations}
\label{fig:er-diagram}
\end{figure}

\section{Entités Principales}

\subsection{Equipment}

L'entité \textbf{Equipment} représente les actifs physiques à maintenir.

\begin{table}[H]
\centering
\caption{Attributs de l'entité Equipment}
\label{tab:equipment-attrs}
\small
\begin{tabularx}{\textwidth}{@{}l l X@{}}
\toprule
\textbf{Attribut} & \textbf{Type} & \textbf{Description} \\
\midrule
id & Integer (PK) & Identifiant unique \\
designation & String(200) & Nom de l'équipement \\
type & String(100) & Type ou catégorie \\
location & String(200) & Emplacement physique \\
status & Enum & Active, Inactive, Maintenance \\
acquisition\_date & Date & Date d'acquisition \\
manufacturer & String(100) & Fabricant \\
serial\_number & String(100) & Numéro de série (unique) \\
\bottomrule
\end{tabularx}
\end{table}

\subsection{Intervention}

L'entité \textbf{Intervention} capture les ordres de travail :

\begin{itemize}
    \item \textbf{Classification} : Type de panne, catégorie, cause
    \item \textbf{Timing} : Date demande, réalisation, durée
    \item \textbf{Coûts} : Coût matériel, main d'œuvre, total
    \item \textbf{Statut} : Open, In Progress, Completed, Closed
\end{itemize}

\subsection{SparePart}

Gère l'inventaire des pièces de rechange : désignation, référence unique, coût unitaire, stock actuel, seuil d'alerte, fournisseur.

\subsection{Technician}

Représente le personnel : nom, email unique, spécialité, taux horaire, statut (Active, Inactive, On Leave).

\section{Tables d'Association}

\begin{itemize}
    \item \textbf{InterventionPart} : N:M interventions-pièces avec quantité
    \item \textbf{TechnicianAssignment} : N:M interventions-techniciens avec heures
    \item \textbf{TechnicianSkill} : N:M techniciens-compétences avec niveau
\end{itemize}

\section{Modèles AMDEC}

\begin{itemize}
    \item \textbf{FailureMode} : Mode de défaillance avec effet et cause
    \item \textbf{RPNAnalysis} : Scores Gravité, Occurrence, Détection (1-10)
\end{itemize}

\section{Modèles RAG}

\begin{itemize}
    \item \textbf{RAGDocument} : Métadonnées des documents indexés
    \item \textbf{RAGDocumentChunk} : Segments de texte pour recherche
    \item \textbf{RAGQuery} : Historique des requêtes
\end{itemize}

\section{Schémas Pydantic}

Les schémas Pydantic assurent la validation :

\begin{itemize}
    \item \textbf{*Base} : Attributs communs
    \item \textbf{*Create} : Schéma de création
    \item \textbf{*Update} : Schéma mise à jour (optionnels)
    \item \textbf{*Response} : Réponse avec id et timestamps
\end{itemize}

\begin{infobox}[Validation Automatique]
Pydantic v2 génère la validation des types, les contraintes de champs, et la documentation OpenAPI automatiquement.
\end{infobox}

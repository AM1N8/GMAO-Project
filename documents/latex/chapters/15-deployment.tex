% ============================================================
% Chapitre 15 : Déploiement
% ============================================================

\chapter{Déploiement}
\label{chap:deployment}

\section{Vue d'Ensemble}

\begin{table}[H]
\centering
\caption{Options de déploiement}
\label{tab:deployment-options}
\small
\begin{tabularx}{\textwidth}{@{}l l X@{}}
\toprule
\textbf{Config} & \textbf{Usage} & \textbf{Caractéristiques} \\
\midrule
Dev local & Dev/Test & Docker Compose, SQLite, hot-reload \\
Staging & Pré-prod & Docker, PostgreSQL, Supabase test \\
Production & Live & K8s/Docker, Supabase prod, CDN \\
\bottomrule
\end{tabularx}
\end{table}

\section{Déploiement Local}

\subsection{Prérequis}

Docker Desktop 4.x+, Node.js 18.x+/pnpm, Python 3.11+, Git.

\subsection{Services Docker Compose}

\begin{itemize}
    \item \textbf{api} : Backend FastAPI (port 8000)
    \item \textbf{ollama} : Serveur LLM (port 11434)
    \item \textbf{redis} : Cache (port 6379)
\end{itemize}

\subsection{Lancement}

Backend : \tech{docker-compose up -d} depuis \file{backend/}. Modèles : \tech{ollama pull llama3.2:3b}. Frontend : \tech{pnpm install} puis \tech{pnpm run dev}.

\section{Configuration Supabase}

Variables frontend (.env.local) : NEXT\_PUBLIC\_SUPABASE\_URL, NEXT\_PUBLIC\_SUPABASE\_ANON\_KEY, NEXT\_PUBLIC\_API\_BASE\_URL.

Variables backend (.env) : DATABASE\_URL, SUPABASE\_JWT\_SECRET, OLLAMA\_BASE\_URL, REDIS\_HOST.

\section{Déploiement Production}

\begin{figure}[H]
\centering
\begin{tikzpicture}[node distance=1.2cm and 1.5cm]
    \node[component, fill=warning!20, minimum width=5cm] (cdn) {CDN (Vercel/Cloudflare)};
    \node[component, below=of cdn, fill=primary!15, minimum width=3cm] (frontend) {Frontend (Vercel)};
    \node[component, below=of frontend, fill=secondary!15, minimum width=3cm] (backend) {Backend (Railway)};
    \node[database, below left=1.2cm and 0.5cm of backend, minimum width=1.5cm] (supabase) {Supabase};
    \node[database, below right=1.2cm and 0.5cm of backend, minimum width=1.5cm] (redis) {Redis};
    
    \draw[arrow] (cdn) -- (frontend);
    \draw[arrow] (frontend) -- (backend);
    \draw[arrow] (backend) -- (supabase);
    \draw[arrow] (backend) -- (redis);
\end{tikzpicture}
\caption{Architecture production}
\label{fig:prod-architecture}
\end{figure}

\section{Migrations Alembic}

\tech{alembic revision --autogenerate -m "desc"} : créer migration. \tech{alembic upgrade head} : appliquer. \tech{alembic downgrade -1} : rollback.

\section{Checklist Déploiement}

\begin{table}[H]
\centering
\caption{Checklist pré-déploiement}
\label{tab:deploy-checklist}
\small
\begin{tabularx}{\textwidth}{@{}X c c@{}}
\toprule
\textbf{Tâche} & \textbf{Dev} & \textbf{Prod} \\
\midrule
Variables d'environnement & \checkmark & \checkmark \\
Base de données migrée & \checkmark & \checkmark \\
Modèles Ollama téléchargés & \checkmark & \checkmark \\
CORS configuré & -- & \checkmark \\
HTTPS activé & -- & \checkmark \\
Monitoring configuré & -- & \checkmark \\
Backup automatique & -- & \checkmark \\
\bottomrule
\end{tabularx}
\end{table}

\begin{warningbox}[Important]
Tester en staging, vérifier sécurité, configurer backups et monitoring avant production.
\end{warningbox}

% ============================================================
% Chapitre 2 : Vue d'Ensemble du Système
% ============================================================

\chapter{Vue d'Ensemble du Système}
\label{chap:overview}

\section{Architecture Globale}

Le système ProAct adopte une architecture \textbf{découplée} (decoupled) avec une séparation nette entre le frontend et le backend, communiquant via une API RESTful sécurisée.

\begin{figure}[H]
\centering
\begin{tikzpicture}[
    node distance=1.2cm and 1.5cm,
    every node/.style={font=\small}
]
    % Frontend Layer
    \node[component, minimum width=7cm, fill=primary!15] (frontend) {Frontend Next.js 14};
    \node[below=0.2cm of frontend, font=\footnotesize\color{gray}] {App Router + React 18 + TypeScript};
    
    % API Gateway
    \node[component, below=1.2cm of frontend, minimum width=3.5cm, fill=secondary!15] (api) {API RESTful};
    \node[below=0.2cm of api, font=\footnotesize\color{gray}] {JWT Auth};
    
    % Backend Layer
    \node[component, below=1.2cm of api, minimum width=7cm, fill=accent!15] (backend) {Backend FastAPI};
    \node[below=0.2cm of backend, font=\footnotesize\color{gray}] {Python 3.11 + SQLAlchemy};
    
    % Services
    \node[service, below left=1cm and -1cm of backend] (kpi) {KPI};
    \node[service, right=0.2cm of kpi] (rag) {RAG};
    \node[service, right=0.2cm of rag] (copilot) {Copilot};
    \node[service, right=0.2cm of copilot] (predict) {Predict};
    
    % Databases
    \node[database, below=2cm of backend, xshift=-2cm] (postgres) {PostgreSQL};
    \node[database, right=1cm of postgres] (faiss) {FAISS};
    \node[database, right=1cm of faiss] (redis) {Redis};
    
    % External
    \node[component, right=2.5cm of backend, minimum width=1.8cm, fill=warning!15, font=\footnotesize] (supabase) {Supabase};
    \node[component, below=1.6cm of supabase, minimum width=1.8cm, fill=warning!15, font=\footnotesize] (ollama) {Ollama};
    
    % Arrows
    \draw[arrow] (frontend) -- (api);
    \draw[arrow] (api) -- (backend);
    \draw[arrow] (backend) -- (kpi);
    \draw[arrow] (backend) -- (rag);
    \draw[arrow] (backend) -- (copilot);
    \draw[arrow] (backend) -- (predict);
    \draw[arrow] (kpi) -- (postgres);
    \draw[arrow] (rag) -- (faiss);
    \draw[arrow] (rag) -- (redis);
    \draw[arrow, dashed] (frontend.east) -- ++(0.8,0) |- (supabase.west);
    \draw[arrow] (copilot.south) |- (ollama.west);
    
\end{tikzpicture}
\caption{Architecture globale du système ProAct}
\label{fig:architecture-globale}
\end{figure}

\section{Flux de Données Principal}

Le système gère plusieurs flux de données interconnectés :

\begin{enumerate}
    \item \textbf{Flux d'authentification} : L'utilisateur s'authentifie via Supabase Auth, qui émet un token JWT contenant les claims de rôle.
    
    \item \textbf{Flux CRUD} : Les opérations sur les entités (équipements, interventions, etc.) transitent par l'API REST protégée.
    
    \item \textbf{Flux analytique} : Les calculs de KPIs sont effectués côté backend et retournés au frontend pour visualisation.
    
    \item \textbf{Flux IA} : Les requêtes RAG et Copilot sont traitées par les services spécialisés utilisant Ollama.
\end{enumerate}

\section{Composants Principaux}

\subsection{Frontend (Next.js)}

Le frontend est construit sur le boilerplate \textbf{Makerkit SaaS Lite} qui fournit :

\begin{itemize}
    \item Authentification intégrée avec Supabase
    \item Structure de navigation multi-tenant
    \item Composants UI modernes (shadcn/ui)
    \item Gestion des thèmes (clair/sombre)
    \item Internationalisation (i18n)
\end{itemize}

\begin{infobox}[Structure des Pages Dashboard]
Les pages du dashboard sont organisées dans \file{apps/web/app/home/} avec des sous-dossiers dédiés : equipment, interventions, spare-parts, technicians, kpi, amdec, assistant, copilot, ocr, ai-forecast.
\end{infobox}

\subsection{Backend (FastAPI)}

Le backend expose une API RESTful complète :

\begin{table}[H]
\centering
\caption{Endpoints API par module}
\label{tab:endpoints}
\small
\begin{tabularx}{\textwidth}{@{}l l X@{}}
\toprule
\textbf{Préfixe} & \textbf{Tag} & \textbf{Description} \\
\midrule
/api/equipment & Equipment & CRUD équipements, stats \\
/api/interventions & Interventions & Ordres de travail \\
/api/spare-parts & Spare Parts & Inventaire pièces \\
/api/technicians & Technicians & Profils, compétences \\
/api/kpi & KPIs & MTBF, MTTR, Disponibilité \\
/api/amdec & AMDEC & Modes de défaillance \\
/api/rag & RAG System & Upload docs, requêtes \\
/api/copilot & Copilot & Assistant conversationnel \\
/api/ocr & OCR Vision & Extraction texte images \\
/api/predict & AI Forecast & Prédiction RUL, MTBF \\
\bottomrule
\end{tabularx}
\end{table}

\subsection{Base de Données}

Le modèle de données relationnelles comprend les entités principales suivantes :

\begin{itemize}
    \item \textbf{Equipment} : Équipements/machines avec métadonnées
    \item \textbf{Intervention} : Ordres de travail avec coûts et durées
    \item \textbf{SparePart} : Pièces de rechange avec niveaux de stock
    \item \textbf{Technician} : Personnel de maintenance avec compétences
    \item \textbf{FailureMode} : Modes de défaillance pour analyse AMDEC
    \item \textbf{RPNAnalysis} : Évaluations RPN (Gravité × Occurrence × Détection)
\end{itemize}

\section{Rôles Utilisateurs}

Le système implémente un modèle RBAC avec quatre niveaux :

\begin{table}[H]
\centering
\caption{Matrice des rôles et permissions}
\label{tab:roles}
\small
\begin{tabularx}{\textwidth}{@{}X c c c c@{}}
\toprule
\textbf{Fonctionnalité} & \textbf{Adm} & \textbf{Sup} & \textbf{Tech} & \textbf{View} \\
\midrule
Créer équipements & \checkmark & \checkmark & -- & -- \\
Modifier interventions & \checkmark & \checkmark & \checkmark & -- \\
Consulter KPIs & \checkmark & \checkmark & \checkmark & \checkmark \\
Gérer techniciens & \checkmark & -- & -- & -- \\
Import/Export données & \checkmark & \checkmark & -- & -- \\
Utiliser Copilot/RAG & \checkmark & \checkmark & \checkmark & \checkmark \\
\bottomrule
\end{tabularx}
\end{table}

\section{Points d'Intégration}

\subsection{Supabase}

L'intégration Supabase fournit :
\begin{itemize}
    \item Authentification (email/mot de passe, OAuth)
    \item Stockage des rôles dans \tech{app\_metadata}
    \item Base de données PostgreSQL managée
\end{itemize}

\subsection{Ollama}

Le serveur Ollama local héberge les modèles LLM pour la génération de réponses RAG, le traitement des requêtes Copilot, et l'extraction OCR avec modèles vision.

\begin{warningbox}[Configuration Requise]
Le serveur Ollama doit être accessible à l'URL configurée dans \file{OLLAMA\_BASE\_URL} (par défaut \tech{http://ollama:11434} en Docker).
\end{warningbox}

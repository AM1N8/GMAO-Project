% ============================================================
% Chapitre 17 : Perspectives d'Évolution
% ============================================================

\chapter{Perspectives d'Évolution}
\label{chap:future}

\section{Feuille de Route}

\begin{figure}[H]
\centering
\begin{tikzpicture}[node distance=0.6cm]
    \node[rectangle, draw=primary, fill=primary!15, minimum width=10cm, minimum height=0.7cm] (v21) {\textbf{v2.1} -- Améliorations IA et Performance};
    \node[rectangle, draw=secondary, fill=secondary!15, minimum width=10cm, minimum height=0.7cm, below=of v21] (v22) {\textbf{v2.2} -- Intégrations Externes};
    \node[rectangle, draw=accent, fill=accent!15, minimum width=10cm, minimum height=0.7cm, below=of v22] (v30) {\textbf{v3.0} -- Multi-tenancy et Enterprise};
\end{tikzpicture}
\caption{Feuille de route}
\label{fig:roadmap}
\end{figure}

\section{Version 2.1 -- Court Terme}

\subsection{Intelligence Artificielle}

\begin{itemize}
    \item \textbf{RAG} : Support PDF natif, recherche hybride, historique multi-tour
    \item \textbf{Prédiction} : Modèles LSTM, saisonnalité, fleet-level analytics
    \item \textbf{OCR} : Support multi-pages, extraction formulaires structurés
\end{itemize}

\subsection{Performance}

Agrégation KPIs pré-calculée, pagination curseurs, compression gzip, intégration Prometheus/Grafana.

\section{Version 2.2 -- Intégrations}

\begin{table}[H]
\centering
\caption{Intégrations prévues}
\label{tab:integrations}
\small
\begin{tabularx}{\textwidth}{@{}l l X@{}}
\toprule
\textbf{Système} & \textbf{Type} & \textbf{Fonctionnalité} \\
\midrule
SAP PM & ERP & Import/sync équipements et OT \\
SCADA/OPC-UA & IoT & Données temps réel machines \\
Microsoft 365 & Productivité & Notifications, calendrier \\
Power BI & Analytics & Export reporting avancé \\
\bottomrule
\end{tabularx}
\end{table}

\subsection{Notifications et API}

Push PWA, alertes email, Slack/Teams, SMS critiques. Endpoint GraphQL, WebSocket temps réel, webhooks sortants, API versioning.

\section{Version 3.0 -- Enterprise}

\subsection{Multi-tenancy}

Isolation données par organisation, personnalisation par tenant, quotas et limites configurables.

\subsection{Sécurité Enterprise}

SSO SAML 2.0, LDAP/AD, MFA obligatoire, audit trail immutable, conformité RGPD.

\subsection{Fonctionnalités Avancées}

\begin{itemize}
    \item \textbf{Workflows} : Designer graphique, approbations multi-niveaux, escalations
    \item \textbf{Reporting} : Générateur personnalisé, dashboards configurables
    \item \textbf{Mobile} : App iOS/Android native, mode hors-ligne, QR codes
\end{itemize}

\section{Évolutions Techniques}

Microservices optionnel, Kubernetes, event sourcing, fine-tuning LLM sur données GMAO, vision AI inspection.

\section{Conclusion}

ProAct constitue une base solide pour une GMAO moderne intégrant l'IA. Les évolutions visent à renforcer les capacités prédictives, faciliter l'intégration, répondre aux exigences enterprise, et améliorer l'expérience utilisateur.

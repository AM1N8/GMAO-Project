% ============================================================
% Chapitre 3 : Architecture Frontend
% ============================================================

\chapter{Architecture Frontend}
\label{chap:frontend}

\section{Vue d'Ensemble}

Le frontend de ProAct est développé avec \textbf{Next.js 14} utilisant le nouvel \textbf{App Router}, basé sur le boilerplate \textbf{Makerkit SaaS Kit Lite}. Cette architecture moderne permet le rendu côté serveur (SSR), la génération statique (SSG) et les Server Components de React.

\begin{techbox}[Caractéristiques Techniques]
\begin{itemize}
    \item \textbf{Framework} : Next.js 14.x avec App Router
    \item \textbf{UI Library} : React 18.x
    \item \textbf{Langage} : TypeScript (strict mode)
    \item \textbf{Styling} : Tailwind CSS v3
    \item \textbf{Composants} : shadcn/ui + Radix UI
    \item \textbf{État} : React Query (TanStack Query)
    \item \textbf{Formulaires} : React Hook Form + Zod
\end{itemize}
\end{techbox}

\section{Structure du Projet}

\subsection{Organisation des Dossiers}

Le projet suit une structure monorepo avec \file{apps/web/} contenant l'application principale. Les dossiers clés sont :

\begin{itemize}
    \item \file{app/} : Pages et layouts (App Router)
    \item \file{app/(marketing)/} : Pages publiques (landing, pricing)
    \item \file{app/auth/} : Authentification (login, signup, callback)
    \item \file{app/home/} : Dashboard protégé avec tous les modules GMAO
    \item \file{components/} : Composants réutilisables
    \item \file{lib/} : Utilitaires, client API, hooks personnalisés
\end{itemize}

\subsection{Pages du Dashboard}

Le dashboard (\file{app/home/}) contient les modules fonctionnels :

\begin{table}[H]
\centering
\caption{Structure des pages dashboard}
\label{tab:pages-dashboard}
\small
\begin{tabularx}{\textwidth}{@{}l X@{}}
\toprule
\textbf{Dossier} & \textbf{Fonctionnalité} \\
\midrule
\file{equipment/} & Gestion des équipements \\
\file{interventions/} & Ordres de travail \\
\file{spare-parts/} & Inventaire pièces de rechange \\
\file{technicians/} & Gestion des techniciens \\
\file{kpi/} & Dashboard KPIs avec graphiques \\
\file{amdec/} & Analyse AMDEC et RPN \\
\file{assistant/} & Interface chat RAG \\
\file{copilot/} & Assistant Copilot \\
\file{ocr/} & Upload et extraction OCR \\
\file{knowledge-base/} & Base de Connaissances (Docs) \\
\file{priority-training/} & Priorisation Formations IA \\
\file{ai-forecast/} & Prédictions maintenance \\
\file{import-export/} & Import/Export CSV \\
\bottomrule
\end{tabularx}
\end{table}

\section{Système de Layout}

\subsection{Layout Principal}

Le fichier \file{app/home/layout.tsx} définit le layout du dashboard avec support pour deux modes de navigation : \textbf{sidebar} (navigation latérale) et \textbf{header} (navigation horizontale). Le mode est déterminé par un cookie utilisateur.

\subsection{Composants de Navigation}

\begin{itemize}
    \item \textbf{HomeSidebar} : Navigation latérale avec icônes et labels
    \item \textbf{HomeMenuNavigation} : Navigation horizontale (mode header)
    \item \textbf{HomeMobileNavigation} : Menu hamburger pour mobile
    \item \textbf{GuidanceWidget} : Widget d'aide IA flottant
\end{itemize}

\section{Client API GMAO}

Le fichier \file{lib/gmao-api.ts} centralise toutes les communications avec le backend. La classe \tech{GmaoApiClient} fournit :

\begin{itemize}
    \item Gestion automatique des tokens JWT via Supabase
    \item Méthodes typées pour chaque entité (Equipment, Intervention, etc.)
    \item Gestion des erreurs avec types personnalisés
    \item Support des paramètres de requête (pagination, filtres)
\end{itemize}

\begin{conceptbox}[Pattern Singleton]
Le client API est exporté comme instance globale \tech{gmaoApi} pour assurer une gestion cohérente des tokens d'authentification à travers l'application.
\end{conceptbox}

\section{Composants UI Réutilisables}

\subsection{Data Tables}

Les tableaux de données utilisent \textbf{TanStack Table} avec les fonctionnalités suivantes :

\begin{itemize}
    \item Tri multi-colonnes avec indicateurs visuels (flèches asc/desc)
    \item Pagination côté client configurable
    \item Filtrage par colonnes avec autocomplétion
    \item Sparklines intégrés pour visualisation rapide des tendances
    \item Actions contextuelles (édition, suppression, détails)
\end{itemize}

\subsection{Formulaires}

Les formulaires sont construits avec \textbf{React Hook Form} pour la gestion d'état et \textbf{Zod} pour la validation. Chaque schéma de validation définit les contraintes qui génèrent automatiquement les messages d'erreur.

\subsection{Graphiques}

Les visualisations sont réalisées avec \textbf{Recharts} : BarChart, LineChart, PieChart, et AreaChart.

\section{Gestion des Thèmes}

Le système supporte les modes clair et sombre via l'attribut \tech{data-theme}. La préférence utilisateur est persistée dans un cookie.

\section{Optimisations de Performance}

\begin{infobox}[Optimisations Implémentées]
\begin{itemize}
    \item \textbf{Server Components} : Réduction du bundle JavaScript client
    \item \textbf{Streaming SSR} : Affichage progressif des composants
    \item \textbf{Image Optimization} : Next/Image avec lazy loading
    \item \textbf{Code Splitting} : Routes dynamiques à la demande
    \item \textbf{React Query Caching} : Mise en cache des requêtes API
\end{itemize}
\end{infobox}

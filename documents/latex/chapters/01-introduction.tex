% ============================================================
% Chapitre 1 : Introduction Générale
% ============================================================

\chapter{Introduction Générale}
\label{chap:introduction}

\section{Contexte et Enjeux}

La maintenance industrielle constitue un pilier fondamental de la performance opérationnelle des entreprises manufacturières. Dans un contexte de compétitivité accrue et de digitalisation croissante, les systèmes de \textbf{Gestion de Maintenance Assistée par Ordinateur} (GMAO) deviennent des outils stratégiques incontournables.

\begin{conceptbox}[Définition GMAO]
Un système GMAO (ou CMMS -- \textit{Computerized Maintenance Management System}) est une solution logicielle permettant de centraliser, planifier, suivre et optimiser l'ensemble des opérations de maintenance d'une organisation.
\end{conceptbox}

Le projet \textbf{ProAct} s'inscrit dans cette dynamique en proposant une plateforme GMAO moderne, enrichie de capacités d'\textbf{Intelligence Artificielle} pour accompagner les équipes de maintenance dans leurs prises de décision.

\section{Objectifs du Projet}

Le système ProAct a été conçu pour répondre aux objectifs suivants :

\begin{enumerate}
    \item \textbf{Centralisation des données} : Regrouper l'ensemble des informations relatives aux équipements, interventions, pièces de rechange et techniciens dans une base de données unifiée.
    
    \item \textbf{Suivi des interventions} : Permettre la création, l'assignation et le suivi complet des ordres de travail avec traçabilité des coûts et temps d'arrêt.
    
    \item \textbf{Analyse des performances} : Calculer et visualiser les KPIs clés de maintenance (MTBF, MTTR, Disponibilité, OEE).
    
    \item \textbf{Aide à la décision par IA} : Intégrer des modules d'intelligence artificielle pour :
    \begin{itemize}
        \item La prédiction des pannes (maintenance prédictive)
        \item L'analyse documentaire via RAG (Retrieval-Augmented Generation)
        \item L'assistance conversationnelle (Copilot)
        \item L'extraction automatique de données (OCR)
    \end{itemize}
    
    \item \textbf{Gestion des compétences} : Suivre les compétences des techniciens et identifier les besoins de formation.
    
    \item \textbf{Analyse AMDEC/RPN} : Évaluer la criticité des modes de défaillance via l'analyse RPN (Risk Priority Number).
\end{enumerate}

\section{Périmètre Fonctionnel}

Le tableau \ref{tab:perimetre} synthétise les principaux modules fonctionnels du système ProAct.

\begin{table}[H]
\centering
\caption{Modules fonctionnels du système ProAct}
\label{tab:perimetre}
\begin{tabular}{@{}lll@{}}
\toprule
\textbf{Module} & \textbf{Fonctionnalités} & \textbf{Utilisateurs} \\
\midrule
Équipements & CRUD, historique, statistiques & Admin, Supervisor \\
Interventions & Gestion ordres de travail & Tous \\
Pièces de rechange & Stock, alertes seuil & Admin, Supervisor \\
Techniciens & Profils, compétences, assignations & Admin \\
KPI Dashboard & MTBF, MTTR, Disponibilité & Tous \\
AMDEC/RPN & Modes de défaillance, criticité & Supervisor, Admin \\
RAG Assistant & Requêtes documentaires IA & Tous \\
Copilot & Assistant conversationnel & Tous \\
OCR & Extraction données images & Admin, Supervisor \\
AI Forecast & Prédiction pannes & Admin, Supervisor \\
\bottomrule
\end{tabular}
\end{table}

\section{Structure de ce Document}

Ce document technique est organisé en plusieurs chapitres couvrant l'ensemble des aspects du système :

\begin{itemize}
    \item \textbf{Chapitres 2-3} : Vue d'ensemble et architecture frontend
    \item \textbf{Chapitre 4} : Système d'authentification et autorisation
    \item \textbf{Chapitres 5-6} : Architecture backend et modèle de données
    \item \textbf{Chapitres 7-8} : Gestion des interventions et soumissions
    \item \textbf{Chapitre 9} : KPIs et tableaux de bord analytiques
    \item \textbf{Chapitre 10} : Modules d'Intelligence Artificielle
    \item \textbf{Chapitres 11-12} : Sécurité et expérience utilisateur
    \item \textbf{Chapitres 13-14} : Performances et tests
    \item \textbf{Chapitres 15-17} : Déploiement, limites et perspectives
\end{itemize}

\section{Technologies Utilisées}

\begin{techbox}[Stack Technologique]
\textbf{Frontend :}
\begin{itemize}
    \item Next.js 14 avec App Router
    \item React 18 + TypeScript
    \item Makerkit SaaS Boilerplate
    \item Tailwind CSS + shadcn/ui
\end{itemize}

\textbf{Backend :}
\begin{itemize}
    \item FastAPI (Python 3.11+)
    \item SQLAlchemy ORM
    \item Pydantic pour la validation
\end{itemize}

\textbf{Base de données :}
\begin{itemize}
    \item PostgreSQL (Supabase)
    \item SQLite (développement local)
\end{itemize}

\textbf{Intelligence Artificielle :}
\begin{itemize}
    \item Ollama (LLM local)
    \item FAISS (recherche vectorielle)
    \item Prophet (séries temporelles)
    \item Scikit-learn, XGBoost
\end{itemize}
\end{techbox}

% ============================================================
% Chapitre 11 : Sécurité
% ============================================================

\chapter{Sécurité}
\label{chap:security}

\section{Vue d'Ensemble}

La sécurité de ProAct est conçue selon le principe de \textbf{défense en profondeur}.

\begin{figure}[H]
\centering
\begin{tikzpicture}[node distance=0.5cm]
    \node[rectangle, draw=primary, fill=primary!10, minimum width=8cm, minimum height=0.7cm, font=\small] (l1) {\textbf{Authentification} -- JWT Supabase};
    \node[rectangle, draw=secondary, fill=secondary!10, minimum width=7cm, minimum height=0.7cm, below=of l1, font=\small] (l2) {\textbf{Autorisation} -- RBAC 4 niveaux};
    \node[rectangle, draw=accent, fill=accent!10, minimum width=6cm, minimum height=0.7cm, below=of l2, font=\small] (l3) {\textbf{Validation} -- Pydantic};
    \node[rectangle, draw=warning, fill=warning!10, minimum width=5cm, minimum height=0.7cm, below=of l3, font=\small] (l4) {\textbf{Transport} -- HTTPS/TLS};
    \node[rectangle, draw=danger, fill=danger!10, minimum width=4cm, minimum height=0.7cm, below=of l4, font=\small] (l5) {\textbf{Données} -- Chiffrement};
\end{tikzpicture}
\caption{Couches de sécurité}
\label{fig:security-layers}
\end{figure}

\section{Authentification JWT}

\begin{itemize}
    \item \textbf{Algorithme} : HS256
    \item \textbf{Secret} : \tech{SUPABASE\_JWT\_SECRET} (256 bits min)
    \item \textbf{Expiration} : 1 heure par défaut
    \item \textbf{Refresh} : Automatique par Supabase
\end{itemize}

\section{Autorisation RBAC}

\begin{table}[H]
\centering
\caption{Hiérarchie des rôles}
\label{tab:role-hierarchy}
\small
\begin{tabularx}{\textwidth}{@{}l l X@{}}
\toprule
\textbf{Rôle} & \textbf{Niveau} & \textbf{Privilèges} \\
\midrule
\role{Admin} & 4 & Accès total, gestion utilisateurs \\
\role{Supervisor} & 3 & Gestion opérationnelle \\
\role{Technician} & 2 & Interventions, soumissions \\
\role{Viewer} & 1 & Consultation seule \\
\bottomrule
\end{tabularx}
\end{table}

\section{Validation des Entrées}

Pydantic valide types et contraintes. SQLAlchemy ORM génère des requêtes paramétrées pour prévenir les injections SQL.

\section{Sécurité des Communications}

Configuration CORS : origines autorisées, méthodes (GET, POST, PUT, DELETE), headers (Authorization, Content-Type).

\begin{warningbox}[Configuration Production]
Remplacer \tech{allow\_origins=["*"]} par la liste explicite des domaines autorisés.
\end{warningbox}

\section{Protection des Données}

\begin{table}[H]
\centering
\caption{Classification des données}
\label{tab:sensitive-data}
\small
\begin{tabularx}{\textwidth}{@{}l l X@{}}
\toprule
\textbf{Donnée} & \textbf{Sensibilité} & \textbf{Protection} \\
\midrule
Mots de passe & Critique & Hash bcrypt (Supabase) \\
Tokens JWT & Haute & HttpOnly cookies \\
Emails & Moyenne & Accès restreint \\
Coûts & Moyenne & Visible selon rôle \\
Données tech. & Basse & Accès authentifié \\
\bottomrule
\end{tabularx}
\end{table}

\section{Checklist Sécurité}

\begin{table}[H]
\centering
\caption{Checklist de sécurité}
\label{tab:security-checklist}
\small
\begin{tabularx}{\textwidth}{@{}X c c@{}}
\toprule
\textbf{Mesure} & \textbf{Impl.} & \textbf{Prod} \\
\midrule
Authentification JWT & \checkmark & \checkmark \\
RBAC 4 niveaux & \checkmark & \checkmark \\
Validation Pydantic & \checkmark & \checkmark \\
Protection SQL Injection & \checkmark & \checkmark \\
Configuration CORS & \checkmark & Configurer \\
HTTPS/TLS & -- & Requis \\
Rate Limiting & -- & Recommandé \\
\bottomrule
\end{tabularx}
\end{table}

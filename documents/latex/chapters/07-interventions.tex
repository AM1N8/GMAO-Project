% ============================================================
% Chapitre 7 : Système d'Interventions
% ============================================================

\chapter{Système d'Interventions}
\label{chap:interventions}

\section{Vue d'Ensemble}

Le module d'interventions constitue le cœur opérationnel du système GMAO. Il gère l'ensemble du cycle de vie des ordres de travail de maintenance.

\begin{conceptbox}[Définition - Intervention]
Une \textbf{intervention} représente une action de maintenance : \textbf{corrective} (suite à panne), \textbf{préventive} (planifiée), ou \textbf{améliorative} (upgrade).
\end{conceptbox}

\section{Cycle de Vie}

\begin{figure}[H]
\centering
\begin{tikzpicture}[
    node distance=1.5cm,
    state/.style={
        rectangle,
        rounded corners=4pt,
        draw,
        minimum width=1.8cm,
        minimum height=0.6cm,
        font=\footnotesize\bfseries
    }
]
    \node[state, fill=warning!30, draw=warning] (open) {OPEN};
    \node[state, fill=primary!30, draw=primary, right=of open] (progress) {IN\_PROGRESS};
    \node[state, fill=success!30, draw=success, right=of progress] (completed) {COMPLETED};
    \node[state, fill=gray!30, draw=darkgray, right=of completed] (closed) {CLOSED};
    \node[state, fill=danger!30, draw=danger, below=1cm of progress] (cancelled) {CANCELLED};
    
    \draw[arrow] (open) -- node[above, font=\tiny] {Assign} (progress);
    \draw[arrow] (progress) -- node[above, font=\tiny] {Fini} (completed);
    \draw[arrow] (completed) -- node[above, font=\tiny] {Valid} (closed);
    \draw[arrow] 
    (open.south) -- ++(0,-0.3) -| ([xshift=3mm]cancelled.west);
    \draw[arrow] (progress.south) -- (cancelled.north);
\end{tikzpicture}
\caption{États du cycle de vie}
\label{fig:intervention-lifecycle}
\end{figure}

\section{Création d'une Intervention}

Via \api{POST /api/interventions/} avec :

\begin{itemize}
    \item \textbf{equipment\_id} (requis) : Identifiant équipement
    \item \textbf{type\_panne} : Classification
    \item \textbf{categorie\_panne} : Catégorie
    \item \textbf{cause} : Cause identifiée
    \item \textbf{date\_demande} : Date de la demande
\end{itemize}

\section{Assignation et Pièces}

L'assignation crée un \tech{TechnicianAssignment} (technicien, heures, taux). L'ajout de pièces via \api{POST /api/interventions/\{id\}/parts} vérifie le stock et décrémente automatiquement.

\section{Requêtes et Filtrage}

\begin{table}[H]
\centering
\caption{Paramètres de filtrage}
\label{tab:intervention-filters}
\small
\begin{tabularx}{\textwidth}{@{}l l X@{}}
\toprule
\textbf{Paramètre} & \textbf{Type} & \textbf{Description} \\
\midrule
skip/limit & int & Pagination \\
status & enum & Filtrer par statut \\
equipment\_id & int & Filtrer par équipement \\
start\_date & date & Début de période \\
end\_date & date & Fin de période \\
type\_panne & string & Recherche par type \\
\bottomrule
\end{tabularx}
\end{table}

\section{Données Capturées}

\begin{table}[H]
\centering
\caption{Données par intervention}
\label{tab:intervention-data}
\small
\begin{tabularx}{\textwidth}{@{}l l X@{}}
\toprule
\textbf{Donnée} & \textbf{Type} & \textbf{Utilisation} \\
\midrule
Durée indisponibilité & Float (heures) & Calcul MTTR \\
Coût matériel & Float (€) & Analyse coûts \\
Coût main d'œuvre & Float (€) & Analyse coûts \\
Type de panne & String & Distribution \\
Date intervention & Date & Tendances \\
\bottomrule
\end{tabularx}
\end{table}

\begin{infobox}[Traçabilité]
Chaque intervention maintient un historique complet : horodatages, transitions de statut, assignations.
\end{infobox}

% ============================================================
% Chapitre 9 : KPI et Analytics
% ============================================================

\chapter{KPI et Analytics}
\label{chap:kpi}

\section{Introduction aux KPIs de Maintenance}

Les \textbf{Key Performance Indicators} (KPIs) sont des métriques essentielles pour évaluer l'efficacité des opérations de maintenance. ProAct implémente les indicateurs standards de l'industrie, calculés automatiquement à partir des données d'intervention.

\section{Indicateurs Principaux}

\subsection{MTBF -- Mean Time Between Failures}

\begin{conceptbox}[Définition MTBF]
Le \kpi{MTBF} (Temps Moyen Entre Pannes) mesure la fiabilité d'un équipement. Plus le MTBF est élevé, plus l'équipement est fiable.

\textbf{Formule :}
\begin{equation}
MTBF = \frac{\sum_{i=1}^{n-1} (t_{i+1} - t_i)}{n - 1}
\end{equation}

Où $t_i$ représente la date de la $i$-ème panne et $n$ le nombre total de pannes.
\end{conceptbox}

Le calcul du MTBF dans ProAct :
\begin{enumerate}
    \item Récupère les interventions correctives triées par date
    \item Calcule les intervalles entre pannes consécutives (en heures)
    \item Retourne la moyenne des intervalles
\end{enumerate}

\textbf{Conditions} : Nécessite au minimum 2 pannes pour calculer un intervalle. Retourne \tech{null} si données insuffisantes.

\subsection{MTTR -- Mean Time To Repair}

\begin{conceptbox}[Définition MTTR]
Le \kpi{MTTR} (Temps Moyen de Réparation) mesure l'efficacité des réparations. Un MTTR bas indique des réparations rapides.

\textbf{Formule :}
\begin{equation}
MTTR = \frac{\sum_{i=1}^{n} D_i}{n}
\end{equation}

Où $D_i$ représente la durée d'indisponibilité de la $i$-ème intervention.
\end{conceptbox}

Le MTTR utilise le champ \tech{duree\_indisponibilite} des interventions, exprimé en heures.

\subsection{Disponibilité}

\begin{conceptbox}[Définition Disponibilité]
La \kpi{Disponibilité} mesure le pourcentage de temps où l'équipement est opérationnel.

\textbf{Formule :}
\begin{equation}
A = \frac{MTBF}{MTBF + MTTR} \times 100 = \frac{T_{op} - T_{down}}{T_{op}} \times 100
\end{equation}
\end{conceptbox}

Le calcul prend en compte :
\begin{itemize}
    \item La période d'analyse (start\_date à end\_date)
    \item Les heures opérationnelles par jour (par défaut 24h)
    \item Les jours opérationnels par semaine (par défaut 7)
    \item Le cumul des durées d'indisponibilité
\end{itemize}

La valeur est bornée entre 0\% et 100\%.

\section{Dashboard KPIs}

L'endpoint \api{GET /api/kpi/dashboard} retourne un ensemble consolidé de métriques en un seul appel :

\begin{table}[H]
\centering
\caption{Structure du Dashboard KPIs}
\label{tab:dashboard-kpis}
\begin{tabular}{@{}ll@{}}
\toprule
\textbf{Catégorie} & \textbf{Métriques} \\
\midrule
Core Metrics & MTBF, MTTR, Disponibilité \\
Counts & Total interventions, équipements actifs \\
Costs & Coût total, matériel, main d'œuvre \\
Distributions & Par type de panne, par statut \\
Trends & Données mensuelles pour graphiques \\
\bottomrule
\end{tabular}
\end{table}

\subsection{Paramètres de Filtrage}

\begin{itemize}
    \item \textbf{start\_date / end\_date} : Période d'analyse
    \item \textbf{equipment\_id} : Filtrer par équipement spécifique
    \item \textbf{operational\_hours\_per\_day} : Heures de fonctionnement (1-24)
\end{itemize}

\section{Distributions et Tendances}

\subsection{Distribution des Pannes}

L'endpoint \api{GET /api/kpi/failure-distribution} retourne le comptage par type de panne avec pourcentages. Permet d'identifier les types de pannes les plus fréquents pour prioriser les actions préventives.

\subsection{Répartition des Coûts}

L'endpoint \api{GET /api/kpi/cost-breakdown} fournit :
\begin{itemize}
    \item Coût matériel total et pourcentage
    \item Coût main d'œuvre total et pourcentage
    \item Coût total de maintenance
\end{itemize}

\section{Visualisation Frontend}

Le dashboard KPI du frontend affiche :

\begin{itemize}
    \item \textbf{Cards métriques} : MTBF, MTTR, Disponibilité avec indicateurs de tendance
    \item \textbf{Graphique en barres} : Distribution des types de pannes
    \item \textbf{Graphique en ligne} : Évolution temporelle des interventions
    \item \textbf{Pie chart} : Répartition des coûts (matériel vs MO)
    \item \textbf{Tableau récapitulatif} : KPIs par équipement
\end{itemize}

\section{Interprétation des KPIs}

\begin{table}[H]
\centering
\caption{Interprétation des KPIs}
\label{tab:kpi-interpretation}
\begin{tabular}{@{}llll@{}}
\toprule
\textbf{KPI} & \textbf{Bon} & \textbf{Acceptable} & \textbf{Critique} \\
\midrule
MTBF & > 500h & 200-500h & < 200h \\
MTTR & < 2h & 2-8h & > 8h \\
Disponibilité & > 95\% & 85-95\% & < 85\% \\
\bottomrule
\end{tabular}
\end{table}

\begin{infobox}[Bonnes Pratiques]
Les seuils ci-dessus sont indicatifs. Il est recommandé de les adapter selon le type d'industrie, la criticité des équipements, et les objectifs de performance définis.
\end{infobox}

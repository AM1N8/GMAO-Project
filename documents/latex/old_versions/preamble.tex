% ============================================================
% ProAct GMAO - Documentation Technique
% Préambule LaTeX - Configuration du document 
% ============================================================

% ============== CONFIGURATION DE BASE ==============
\documentclass[12pt, a4paper, openany]{report}

% Encodage et langue
\usepackage[utf8]{inputenc}
\usepackage[T1]{fontenc}
\usepackage[french]{babel}
\usepackage{amssymb}

% ============== POLICES ET TYPOGRAPHIE ==============
% Utilisation de Palatino (mathpazo) pour le corps et Helvetica pour les titres
\usepackage{mathpazo} 
\usepackage[scaled=0.95]{helvet}
\usepackage{courier} % Pour le code
\usepackage{microtype} % Amélioration micro-typographique

% ============== MISE EN PAGE ==============
\usepackage[
    top=2cm,
    bottom=2cm,
    left=2cm,
    right=2cm,
    headheight=26pt,
    footskip=1cm
]{geometry}

% Interligne optimisé (compact)
\usepackage{setspace}
\setstretch{1.1}
\setlength{\parskip}{0.4em}
\setlength{\parindent}{0pt}

% Optimisation des placements de figures pour éviter les vides
\renewcommand{\topfraction}{0.85}
\renewcommand{\bottomfraction}{0.85}
\renewcommand{\textfraction}{0.15}
\renewcommand{\floatpagefraction}{0.75}

% ============== COULEURS PROFESSIONNELLES ==============
\usepackage[dvipsnames,svgnames,table]{xcolor}

% Palette "ProAct" Modernisée
\definecolor{primary}{HTML}{0052CC}      % Royal Blue - Professionnel et Tech
\definecolor{secondary}{HTML}{008DA6}    % Teal - Frais et moderne
\definecolor{accent}{HTML}{6554C0}       % Deep Purple - Distinctif
\definecolor{darkgray}{HTML}{2D3748}     % Slate Gray - Texte principal doux
\definecolor{mediumgray}{HTML}{4A5568}   % Texte secondaire
\definecolor{lightgray}{HTML}{EDF2F7}    % Fonds très clairs
\definecolor{codebg}{HTML}{F7FAFC}       % Fond blocs de code
\definecolor{bordergray}{HTML}{CBD5E0}   % Bordures subtiles
% Couleurs sémantiques
\definecolor{warning}{HTML}{DD6B20}      % Orange brûlé (moins agressif)
\definecolor{danger}{HTML}{E53E3E}       % Rouge élégant
\definecolor{success}{HTML}{38A169}      % Vert naturel

% ============== GRAPHIQUES ==============
\usepackage{graphicx}
\usepackage{float}
\usepackage{wrapfig}
\usepackage{tikz}
\usetikzlibrary{
    shapes.geometric,
    shapes.arrows,
    arrows.meta,
    positioning,
    shadows,
    backgrounds,
    calc,
    fit
}
\graphicspath{{figures/}}

% Styles TikZ personnalisés
\tikzset{
    component/.style={
        rectangle,
        rounded corners=5pt,
        draw=primary,
        fill=primary!10,
        minimum width=3cm,
        minimum height=1cm,
        text centered,
        font=\small\bfseries,
        drop shadow
    },
    database/.style={
        cylinder,
        cylinder uses custom fill,
        cylinder body fill=secondary!20,
        cylinder end fill=secondary!40,
        draw=secondary,
        shape border rotate=90,
        minimum width=2cm,
        minimum height=1.5cm,
        text centered,
        font=\small\bfseries,
        drop shadow
    },
    service/.style={
        rectangle,
        rounded corners=3pt,
        draw=accent,
        fill=accent!10,
        minimum width=2.5cm,
        minimum height=0.8cm,
        text centered,
        font=\footnotesize
    },
    arrow/.style={
        ->,
        >=Stealth,
        thick,
        color=darkgray
    }
}

% ============== TABLEAUX ==============
\usepackage{booktabs}   % Lignes de tableau professionnelles (toprule, midrule, bottomrule)
\usepackage{longtable}
\usepackage{array}
\usepackage{multirow}
\usepackage{colortbl}
\usepackage{tabularx}
\usepackage{adjustbox}

% Types de colonnes personnalisés
\newcolumntype{L}[1]{>{\raggedright\arraybackslash}p{#1}}
\newcolumntype{C}[1]{>{\centering\arraybackslash}p{#1}}
\newcolumntype{R}[1]{>{\raggedleft\arraybackslash}p{#1}}
\renewcommand{\arraystretch}{1.4} % Tableaux plus aérés

% ============== ENCADRÉS (TCOLORBOX) ==============
\usepackage[most]{tcolorbox}

% Style global pour les boîtes
\tcbset{
    enhanced,
    sharp corners=flt, % Coins légèrement arrondis
    colback=white,
    boxrule=0.5pt,
    drop shadow={opacity=0.1, shadow xshift=2pt, shadow yshift=-2pt},
    left=10pt, right=10pt, top=8pt, bottom=8pt,
    fonttitle=\bfseries\sffamily
}

% Boîte d'information (Bleu)
\newtcolorbox{infobox}[1][Info]{
    colframe=primary,
    colbacktitle=primary,
    title={#1},
    colback=primary!5!white
}

% Boîte d'avertissement (Orange)
\newtcolorbox{warningbox}[1][Attention]{
    colframe=warning,
    colbacktitle=warning,
    title={#1},
    colback=warning!5!white
}

% Boîte technique (Gris)
\newtcolorbox{techbox}[1][Détails Techniques]{
    colframe=darkgray,
    colbacktitle=darkgray,
    title={#1},
    colback=darkgray!5!white
}

% Boîte concept clé (Violet)
\newtcolorbox{conceptbox}[1][Concept Clé]{
    colframe=accent,
    colbacktitle=accent,
    title={#1},
    colback=accent!5!white
}

% ============== CODE SOURCE ==============
\usepackage{listings}

% Style général du code
\lstdefinestyle{codestyle}{
    basicstyle=\ttfamily\small\color{darkgray},
    commentstyle=\itshape\color{mediumgray},
    keywordstyle=\bfseries\color{primary},
    stringstyle=\color{success},
    numberstyle=\tiny\color{bordergray},
    numbers=left,
    numbersep=10pt,
    backgroundcolor=\color{codebg},
    frame=ltrb,
    framerule=0pt, % Géré par tcolorbox si besoin, ou simple ici
    framesep=5pt,
    rulecolor=\color{lightgray},
    breaklines=true,
    breakatwhitespace=false,
    showspaces=false,
    captionpos=b,
    keepspaces=true,
    tabsize=2,
    xleftmargin=15pt,
    framexleftmargin=10pt,
    aboveskip=1em,
    belowskip=1em
}

\lstset{style=codestyle}

% Langages spécifiques
\lstdefinelanguage{TypeScript}{
    keywords={typeof, new, true, false, catch, function, return, null, catch, switch, var, if, in, while, do, else, case, break, const, let, interface, type, extends, implements, import, export, from, async, await},
    sensitive=true,
    comment=[l]{//},
    morecomment=[s]{/*}{*/},
    morestring=[b]',
    morestring=[b]"
}

% ============== EN-TÊTES ET PIEDS DE PAGE ==============
\usepackage{fancyhdr}
\pagestyle{fancy}
\fancyhf{}

% En-tête : Chapitre à gauche, Numéro page à droite
\fancyhead[L]{\sffamily\color{mediumgray}\nouppercase{\leftmark}}
\fancyhead[R]{\sffamily\bfseries\color{primary}\thepage}

% Pied de page : Titre doc et auteur
\fancyfoot[C]{\sffamily\footnotesize\color{mediumgray}ProAct GMAO -- Documentation Technique v2.0}

\renewcommand{\headrulewidth}{0.5pt}
\renewcommand{\footrulewidth}{0pt}
\renewcommand{\headrule}{\hbox to\headwidth{\color{primary}\leaders\hrule height \headrulewidth\hfill}}

% Style Plain (pour début chapitre)
\fancypagestyle{plain}{
    \fancyhf{}
    \fancyfoot[C]{\sffamily\color{darkgray}\thepage}
    \renewcommand{\headrulewidth}{0pt}
}

% ============== TITRES DES CHAPITRES ==============
\usepackage{titlesec}

% Format Chapitre : Numéro géant gris + Titre bleu + Ligne
\titleformat{\chapter}[display]
    {\sffamily\bfseries\color{primary}} % Format label
    {\filleft\fontsize{60}{60}\selectfont\color{lightgray}\thechapter} % Label (Numéro)
    {-15pt} % Sep
    {\titlerule[1pt]\vspace{1ex}\filright\Huge} % Before title

% Format Section : Souligné
\titleformat{\section}
    {\sffamily\Large\bfseries\color{darkgray}}
    {\thesection}
    {1em}
    {}

% Format Subsection : Simple
\titleformat{\subsection}
    {\sffamily\large\bfseries\color{secondary}}
    {\thesubsection}
    {1em}
    {}

% Espacement compact
\titlespacing*{\chapter}{0pt}{-50pt}{20pt}
\titlespacing*{\section}{0pt}{12pt}{6pt}
\titlespacing*{\subsection}{0pt}{8pt}{4pt}

% ============== TABLE DES MATIÈRES ==============
\usepackage{tocloft}
\renewcommand{\cftchapfont}{\sffamily\bfseries\color{primary}}
\renewcommand{\cftsecfont}{\sffamily\color{darkgray}}
\renewcommand{\cftchappagefont}{\bfseries\color{darkgray}}

% ============== HYPERLIENS ==============
\usepackage[
    hidelinks, % Pas de boîtes rouges moches
    colorlinks=true,
    linkcolor=primary,
    urlcolor=secondary,
    citecolor=accent,
    pdfauthor={Mohamed Amine Darraj},
    pdftitle={ProAct GMAO - Documentation Technique},
    breaklinks=true
]{hyperref}

% ============== COMMANDES PERSONNALISÉES ==============
% Style pour les items de liste
\usepackage{enumitem}
\setlist{nosep}
\setlist[itemize]{label=\color{primary}$\bullet$, leftmargin=*}
\setlist[enumerate]{font=\bfseries\color{secondary}, leftmargin=*}

% Macros sémantiques
\newcommand{\tech}[1]{\texttt{\color{primary}\textbf{#1}}}
\newcommand{\file}[1]{\texttt{\color{mediumgray}#1}}
\newcommand{\api}[1]{\texttt{\color{accent}#1}}
\newcommand{\kpi}[1]{\textbf{\color{success}#1}}
\newcommand{\role}[1]{\textit{\color{secondary}#1}}
